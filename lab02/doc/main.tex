
%\documentclass{report}
\documentclass{article}
%\documentclass{scrartcl}

%\usepackage{mslapa}
\usepackage{hyperref}
\usepackage{amsmath}
\usepackage{graphicx}
\usepackage{ulem}
\usepackage{vmargin}
\usepackage{tabularx}
\usepackage{sectsty}
\usepackage{pbox}
\usepackage{bigstrut}
\usepackage{enumerate}
\usepackage{listings}
\usepackage{parskip}   % space paragraphs but dont indent
\usepackage{verbatim}  % \verbatiminput{file}
\usepackage{gensymb}
\usepackage{color}

\usepackage{natbib}

%\setpapersize{USletter}
%\sectionfont{\normalsize}
%\subsectionfont{\normalsize}
\setmarginsrb{1.0in}{1.0in}{1.0in}{1.0in}{0in}{0.25in}{0in}{0.20in}

% configure \bigstrut size
% This configures spacing above and below rows in a tabularx.
\renewcommand{\bigstrutjot}{2.0\jot}

\providecommand{\e}[1]{\ensuremath{\times 10^{#1}}}

\raggedright

\begin{document}

% {{{ Title Page

\title{Lab 2 (EECE 344)}
\date{April 3, 2012}
\author{Jeremiah Mahler}

\maketitle
% }}}

\tableofcontents

\pagebreak

% {{{ Introduction
\section{Introduction}

The purpose of the "device" described in this document is mainly
to orient a user with some of the basic facilities of a STM32L Discovery\citep{UM1079}
(ARM) board and a Lattice CPLD\cite{EB66}
\footnote{For the remainder of this document the term "CPLD" will
be used to refer to the CPLD board and "ARM" will be used to
refer to the ARM board}.
It is somewhat of a Rube Goldberg machine for electrical engineers
in that it does neat things but it does not do much that is useful.

The goal of this document is to describe the "device" in enough
detail such that a person could re-create the system if given
an identical parts.
This document is not an instruction manual and therefore does not
describe each step that would be required.
It also does not describe tasks such as how to solder or wire wrap.

The general operation of this device is fairly simple.
The ARM board is clocked and contains a cpu so it can do the processing
tasks.
The CPLD does not use a clock so it can only do logic or sequential tasks.
These two devices communicate using SPI with the ARM as the master and
the CPLD as the slave.

The CPLD acts as an interface.
A bank of switches are used to enter 8-bits of data.
This data is sent to the ARM when requested.
And any data it receives (during the full duplex SPI transaction)
is displayed on to an external bank of LEDs.

The ARM board does more useful computations.
The 8-bit data it receives is treated as two 4-bit signed numbers
in 2's compliment form.
These two numbers can be added or subtracted from each other.
A user input button provides the switch, when released they are
added, when depressed they are subtracted.
Whether this operation resulted in an overflow and whether or
not the result is signed must also be accounted for.
Then this result is sent to two different places.

The first place the result is sent is back to the CPLD to be
displayed on the external LEDs.
The second place is on to LCD on board the ARM.

The specific details of this general plan will be described 
in the remainder of this document.

% TODO - picture of board

% }}}

% {{{ Pin Assignments
\section{Pin Assignments}
\label{sec:pa}

One of the critical design specifications of this device are
the pin assignments.
Table \ref{tbl:pins} lists the connections that were used.
Keep in mind that these assignments are specific to this
design and their choice was somewhat arbitrary.
Other pins, such as those for the SPI on the ARM (PA5, PA12, PA11),
cannot be changed due conflicts with other resources on the board.

\begin{table}
	\center
	\begin{tabular}{| l | l | l | l | l | l | l |}
		\hline
		\multicolumn{1}{| c |}{Verilog} &
		\multicolumn{1}{| c |}{} &
		\multicolumn{1}{| c |}{ARM} &
		\multicolumn{4}{| c |}{CPLD} \\
		\hline
		name & description & pin  &  function & Mach XO Ball & Header & Pin \\
		\hline
		SCLK & SPI clock & PA5 & PT9B & D7 & J9 & 11 \\
		SS\_L & SPI slave select & PB5 & PR4C & F13 & J7 & 1 \\
		MOSI & SPI master out slave in & PA12 & PR4D & F12 & J7 & 3 \\
		MISO & SPI master in slave out & PA11 & PR5C  & B16 & J7 & 5 \\
		     & ground                  & GND  &       &     & J8 & 5 \\
		\hline
		\multicolumn{1}{| c |}{Verilog} &
		\multicolumn{1}{| c |}{} &
		\multicolumn{1}{| c |}{input switches} &
		\multicolumn{4}{| c |}{CPLD} \\
		\hline
		name & description & pin  &  function & Mach XO Ball & Header & Pin \\
		\hline
		in\_sw[0] & input switch 8 & 8 & PT2C & B2 & J5 & 1 \\
		in\_sw[1] & input switch 7 & 7 & PT9A & D8 & J5 & 2 \\
		in\_sw[2] & input switch 6 & 6 & PT2D & B3 & J5 & 3 \\
		in\_sw[3] & input switch 5 & 5 & PT9C & E8 & J5 & 4 \\
		in\_sw[4] & input switch 4 & 4 & PT3A & A2 & J5 & 5 \\
		in\_sw[5] & input switch 3 & 3 & PT9D & E9 & J5 & 6 \\
		in\_sw[6] & input switch 2 & 2 & PT3B & A3 & J5 & 7 \\
		in\_sw[7] & input switch 1 & 1 & PT10A & A10 & J5 & 8 \\
		          & power, Vdd, pull up &      &     &    & J9 & 1 \\
		          & ground &  & & & J6 & 2 \\
		\hline
		\multicolumn{1}{|c|}{Verilog} &
		\multicolumn{1}{|c|}{} &
		\multicolumn{1}{|c|}{output LEDs} &
		\multicolumn{4}{|c|}{CPLD} \\
		\hline
		name & description & pin  &  function & Mach XO Ball & Header & Pin \\
		\hline
%		led\_ext[10] & output led 10 & 10 & PT5C & B4 & J5 & 21 \\
%		led\_ext[9] & output led 9 & 9 & PT12A & A11  & J5 & 22 \\
		led\_ext[0] & output led 8 & 8 & PT15D & B5   & J5 & 23 \\
		led\_ext[1] & output led 7 & 7 & PT12B & A12  & J5 & 24 \\
		led\_ext[2] & output led 6 & 6 & PT6E  & E7   & J5 & 25 \\
		led\_ext[3] & output led 5 & 5 & PT12C & B11  & J5 & 26 \\
		led\_ext[4] & output led 4 & 4 & PT6F  & E6   & J5 & 27 \\
		led\_ext[5] & output led 3 & 3 & PT12D & B12  & J5 & 28 \\
		led\_ext[6] & output led 2 & 2 & PT16C & A5   & J5 & 29 \\
		led\_ext[7] & output led 1 & 1 & PT13C & C11  & J5 & 30 \\
		          & ground &  & & & J6 & 16 \\
		\hline
	\end{tabular}
	\caption{Definition of the pin assignments between the ARM board,
		the CPLD, and other devices.
        Notice that the switch and LEDs are reversed.
        This was done so that the orientation from LSB to MSB is from
        right to left.
        }
	\label{tbl:pins}
\end{table}

% describe the pin meanings
Deciphering these pin assignments can be somewhat confusing.
On the ARM board there are headers on either side with each
pin clearly labeled.
On the CPLD the pins correspond to headers (J9, J7, etc)
and pins within those headers\citep[Pg. 11-14]{EB66}.
The header and pin numbers are printed at the end of each header.
The "Mach XO Ball" is used to identify the location on the
chip itself.
This is used for assigning pins when configuring the CPLD
using Diamond\cite{Diamond}.

% configuring using Diamond
Configuring the CPLD pins using Diamond\cite{Diamond} can
be confusing with the bewildering array of options and settings.
Luckily the requirements for this device are fairly simple.
All pins should be configured for low voltage 3.3 volt CMOS.
Output pins can be configured for 8 mA.
And no pull up or pull down options are required.

% ARM to Lattice Diamond
For the pins used with for the SPI that connect between the
ARM and the CPLD there are no special requirements.
They are both designed to work with 3.3 volt CMOS voltage levels.

% ground
For the ground connections each board provides several connection
points.
In general any connection to ground should work and
it is not required to connect to a specific ground pin.
For example the ARM board has four pins labeled GND.
Any of these should work equally well.
It is also a good idea to connect the grounds together so
that they all reference the same common ground.

% input switches
The input switches to the CPLD can be interfaced by connecting
one end to ground and the other end to the pin along with a
pull up resistor to Vdd.
A resistor value between 1k and 10k should be acceptable.
And the pull up voltage for Vdd can be sourced from a pin on
the board (Table \ref{tbl:pins}).

% output leds
The output LEDs are connected to the CPLD using a series
resistor.
Vdd would connect to the resistor which connects to the led
(forward biased) which connects to the pin.
The value of the resistor should limit the current to approximately
10 mA.
A value of 300 $\ohm$ is a typical value.

% reset pin
To reset the boards their reset pins must be configured.
The ARM board provides a \verb+NRST+ pin which is active
low when the reset button is pushed\cite[Pg. 17, 20]{UM1079}.
This can then be connected to the CPLD to cause
it to reset using \verb+GSRN+\citetext{\citealp[Pg. 46, 50]{DS1002}; \citealp[Pg. 8]{EB66}}.

% }}}

% {{{ SPI
\section{Serial Peripheral Interface (SPI)}

In order for the ARM and CPLD to communicate using SPI\citetext{ \citealp[Pg. 278]{cady2009microcontrollers}; \citealp[Pg. 665]{STRM0038}}
they must both have the same options set.
Table \ref{tbl:spi} lists the options that were used.

\begin{table}
\center
\begin{tabular}{|l|l|}
	\hline
	option & value \\
	\hline
	MSB & first \\
	CPOL & 0 \\
	CPHA & 0 \\
	SS\_L & slave select on low \\
	\hline
\end{tabular}
\caption{SPI configuration options}
\label{tbl:spi}
\end{table}

The MSB option specifies that the most significant bit is sent
out first.
CPOL specifies the polarity as zero and CPHA specifies the phase
as zero (first clock edge).
SS\_L specifies that slave is selected when this line is low.

The pins used for the SPI on the ARM board were chosen because
they did not conflict with any other on board devices.
This resulted in pins PA5, PA12 and PA11 with SPI1 for SCK, MOSI,
and MISO respectively\cite[Pg. 24]{UM1079}

Since this device does not require high speed it was
chosen to limit the speeds in an effort to improve reliability.
During testing the value of the SPI clock was found to be approximately
60 kHz.
Since the CPLD is driven by this clock and does not require a specific
frequency it may work with other values.
Certainly values within the same order of magnitude are likely to work.

The ARM, as the SPI master, controls when an SPI transaction occurs.
The simplest way to control this, as was done in this design, is to use polling.
This method is inefficient since data is repeatedly being sent
whether it needs to or not
but it has the side effect of helping to mitigate errors.
Since each new transaction resets the registers to the beginning,
any erroneous data will be quickly overwritten with new data.
This will work as long as the number of transactions without errors
far exceed the number of transactions with errors.

% }}}

% {{{ CPLD input/output formats
\section{CPLD input/output formats}
\label{sec:cpld}

As was discussed previously, the CPLD has switch inputs
and LED outputs.
The orientation and format of this bits is the subject of
this section.

The bank of switches are typically numbered.
In this case they were numbered one through eight.
The common orientation is from right to left starting
with the LSB.
This orientation was applied here but the result is
largest switch number corresponds to the LSB (reversed).

The position of the output LEDs mimic thos of the input
switches with the LSB on the right most side.
% }}}

% {{{ Calculations
\section{Calculations}
\label{sec:calc}

The bit positions described here are referenced using the
most significant bit (MSB) and least significant bit (LSB).
The orientation from left to right does not necessarily
correspond to the orientation of the switches and LEDs.
Refer to Section \ref{sec:pa} for a description of the orientation.

The are several steps that must be performed in order for
the ARM to perform the necessary calculations.
Each will be discussed below.

First the data must be received.
The data should arrive over the SPI with the MSB first.
And this needs to be split in to two 4-bit segments
as is shown below.

\begin{verbatim}
MSB         LSB
8 7 6 5 4 3 2 1     8-bit data
   /        \
4 3 2 1   4 3 2 1   4-bit segments
   B         A      numbers A and B
\end{verbatim}

But the ARM does not have a 4-bit data type.
In order to solve this problem bit masks can be used to
translate the data in to the required size.
For example if the size necessary is 32-bits.
This data is first set to zero, then logically OR'ed with
the incoming data to "transfer" the bits.

Next it must be chosen which operation to perform, add or subtract.
To decide which the on board user push button\cite[Pg. 17]{UM1079}
is used.  When it is released addition is performed, when it
is pressed, subtraction is performed.
When subtraction is performed the second number (LSB section)
is subtracted from the first (MSB section).

% 2's compliment

Another design requirement is that the add and subtract operations
set the overflow and carry bits.
To accomplish this using ARM specific assembly instructions
must be used \cite[Pg. 54]{furber2000arm} and a status
register must be examined\cite[Pg. 40]{furber2000arm}.

Since the 4-bits had to be placed in to a larger data type
they should be shifted to towards the MSB.
This will ensure that the operation will set the overflow and
sign bit appropriately.

Once the addition or subtraction has been performed and the
results of status register are known this data must be
manipulated in to two different formats, one for the LCD,
and the other for the LEDs.

The format for the LEDs is shown below.
\begin{verbatim}
     LEDS
MSB         LSB
8 7 6 5 4 3 2 1
---------------
    N V 4 3 2 1 
\end{verbatim}
Where \verb+N+ is 1 if it is negative and 0 if positive.
\verb+V+ is 1 if overflow occurred and 0 otherwise.
And positions 4 through 1 represent the unsigned 4-bit result
with the MSB at position 4.

Some examples of the format of the LCD are shown below.
\begin{verbatim}
N1V0-3
N0V1+6
N0V0 0
\end{verbatim}
Notice that \verb+N+ and \verb+V+ precede the bit which
indicates the negative flag or overflow respectively.
The remaining space is used to display the result in decimal including the sign.
A negative zero is disallowed.

Table \ref{tbl:calc} list several test cases showing the expected
outputs on the LEDs and the LCD when given certain switch inputs
and whether or not the user button is pressed.
This list is far from exhaustive.
Given that the input switches have 8-bits and for each of
these bits there are two possibilities (button pressed/released)
this results in 512 possible combinations ($2 \cdot 2^8$).

\begin{table}
\begin{tabular}{|c c|c c|c c|}
\hline
\multicolumn{2}{|c}{\textbf{switches}} & \multicolumn{2}{|c|}{\textbf{LEDs}}       & \multicolumn{2}{c|}{\textbf{LCD}} \\
A   & B                        & button=pressed & button=released & button=pressed & button=released \\
\hline
0000 & 0000 & 00 0000 & 00 0000 & N0V0 0 & N0V0 0 \\
0000 & 0001 & 10 1111 & 00 0001 & N1V0-1 & N0V0+1 \\
1010 & 0011 & 01 0111 & 10 1101 & N0V1+7 & N1V0-3 \\
1100 & 0100 & 00 0000 & 10 1000 & N1V0-8 & N0V0 0 \\
1100 & 0110 & 01 0110 & 00 0010 & N0V1+6 & N0V0+2 \\
1111 & 1000 & 00 0111 & 01 0111 & N0V0+7 & N0V1+7 \\
\hline
\end{tabular}
\caption{Calculation test values showing switch inputs, LED, and LCD outputs.
The orientation used here has the MSB on the left and the LSB on the right.
For the subtraction operation the results equals A - B.
When the button is released addition is performed, when it is pressed subtraction is performed.}
\label{tbl:calc}
\end{table}

% }}}

% {{{ References
\clearpage

\pagebreak
\renewcommand*{\refname}{\vspace{-8mm}}
\section{References}
%\bibliographystyle{ieeetr}
\bibliographystyle{plain}
\bibliography{references}
% }}}

\end{document}

% vim:foldmethod=marker

